\documentclass[a4paper,11pt,final]{article}
% Pour une impression recto verso, utilisez plutôt ce documentclass :
%\documentclass[a4paper,11pt,twoside,final]{article}

\usepackage[english,francais]{babel}
\usepackage[utf8]{inputenc}
\usepackage[T1]{fontenc}
\usepackage[pdftex]{graphicx}
\usepackage{setspace}
\usepackage{hyperref}
\usepackage[french]{varioref}
\usepackage{amsmath,amsfonts,amssymb}
\newcommand{\reporttitle}{Rapport}     % Titre
\newcommand{\reportauthor}{Yassine \textsc{Kened} \$ Adrien Bermudez \$ Pierre-Louis Nack} % Auteur
\newcommand{\reportsubject}{Etude d'articles} % Sujet
\newcommand{\HRule}{\rule{\linewidth}{0.5mm}}
\setlength{\parskip}{1ex} % Espace entre les paragraphes

\hypersetup{
    pdftitle={\reporttitle},%
    pdfauthor={\reportauthor},%
    pdfsubject={\reportsubject},%
    pdfkeywords={rapport} {vos} {mots} {clés}
}

\begin{document}
  \begin{titlepage}
  \begin{sffamily}
  \begin{center}

    % Upper part of the page. The '~' is needed because \\
    % only works if a paragraph has started.
   % \includegraphics[scale=0.04]{}~\\[1.5cm]

    \textsc{\LARGE École Nationale Supérieure d'Informatique pour l'Industrie et l'Entreprise}\\[3cm]

    \textsc{\Large Rapport }\\[5cm]

    % Title
    \HRule \\[0.4cm]
    { \huge \bfseries Contrôle stochastique\\[0.4cm] }

    \HRule \\[6cm]
  %  \includegraphics[scale=0.2]{}
    

    % Author and supervisor
    \begin{minipage}{0.4\textwidth}
      \begin{flushleft} \large
      
        Yassine \textsc{Kened}\\
        Pierre-Louis nack\\
        Adrien Bermudez \\
        
      \end{flushleft}
    \end{minipage}
    \begin{minipage}{0.4\textwidth}
      \begin{flushright} \large
        \emph{Tuteur :} M.\textsc{Vathana Lyvath}\\
      \end{flushright}
    \end{minipage}

    \vfill

    % Bottom of the page
    {\large 1\ier{} Décembre 2015-Janvier 2016}

  \end{center}
  \end{sffamily}
\end{titlepage}

  \cleardoublepage % Dans le cas du recto verso, ajoute une page blanche si besoin
  \tableofcontents % Table des matières
  \sloppy          % Justification moins stricte : des mots ne dépasseront pas des paragraphes

  \cleardoublepage
  \section*{Introduction} % Pas de numérotation
\addcontentsline{toc}{section}{Introduction} % Ajout dans la table des matières

  \cleardoublepage
  \section{La première section}


\subsection{Une sous section}








  \cleardoublepage
  \section{Partie2}
\label{p2}




  \cleardoublepage
  \section*{Conclusion}
\addcontentsline{toc}{section}{Conclusion}

Pour conclure, 

  \cleardoublepage
  \phantomsection\addcontentsline{toc}{section}{Références}
\begin{thebibliography}{ABC}	
    \bibitem[REF]{reference} auteur. \emph{titre}. édition, année.
    \bibitem[LPP]{lpp} Rolland. \emph{LaTeX par la pratique}. O'Reilly, 1999.
\end{thebibliography}

\end{document}
