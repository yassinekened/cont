\documentclass[a4paper,11pt,final]{article}
% Pour une impression recto verso, utilisez plutôt ce documentclass :
%\documentclass[a4paper,11pt,twoside,final]{article}

\usepackage[english,francais]{babel}
\usepackage[utf8]{inputenc}
\usepackage[T1]{fontenc}
\usepackage[pdftex]{graphicx}
\usepackage{setspace}
\usepackage{hyperref}
\usepackage[french]{varioref}
\usepackage{amsmath,amsfonts,amssymb}
\newcommand{\reporttitle}{Rapport}     % Titre
\newcommand{\reportauthor}{Yassine \textsc{Kened} \$ Adrien Bermudez \$ Pierre-Louis Nack} % Auteur
\newcommand{\reportsubject}{Etude d'articles} % Sujet
\newcommand{\HRule}{\rule{\linewidth}{0.5mm}}
\setlength{\parskip}{1ex} % Espace entre les paragraphes

\hypersetup{
    pdftitle={\reporttitle},%
    pdfauthor={\reportauthor},%
    pdfsubject={\reportsubject},%
    pdfkeywords={rapport} {vos} {mots} {clés}
}

\begin{document}
  \include{title}
  \cleardoublepage % Dans le cas du recto verso, ajoute une page blanche si besoin
  \tableofcontents % Table des matières
  \sloppy          % Justification moins stricte : des mots ne dépasseront pas des paragraphes

  \cleardoublepage
  \section*{Introduction} % Pas de numérotation
\addcontentsline{toc}{section}{Introduction} % Ajout dans la table des matières

  \cleardoublepage
  \section{La première section}


\subsection{Une sous section}








  \cleardoublepage
  \section{Partie2}
\label{p2}




  \cleardoublepage
  \section*{Conclusion}
\addcontentsline{toc}{section}{Conclusion}

Pour conclure, 

  \cleardoublepage
  \include{references}
\end{document}
